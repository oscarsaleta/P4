\documentclass[a4paper,10pt]{article}
\usepackage{listings}
\usepackage{color}
%
% TO BE INCLUDED IN THE PREAMBLE OF THE TEX FILE WITH \INPUT{MAPLEDEF}.
%
%

\definecolor{maplebgcolor}{rgb}{0.97,.97,.97}

\lstdefinelanguage{maple}
    {%
    morestring=[b]",
    morecomment=[l]\#,
    sensitive=false,
    morekeywords=[1]{proc,local,global,end,for,from,to,by,do,%
            od,if,else,fi,and,or,then,not,restart,type,return,%
    minus,union,read,save,set,list,member},
    morekeywords=[2]{simplify,normal,expand,expanded,evalf,map,collect,diff,%
            subs,eval,op,nops,coeff,rational,float,even,odd,convert,degree,sign,(,),[,]}}

\lstset{%
    basicstyle=\small,
    keywordstyle=[1]\bfseries,
    keywordstyle=[2]\underbar,
    identifierstyle=\slshape,
    firstnumber=auto,
    numbers=left,
    commentstyle=,
    stringstyle=\ttfamily,
    showstringspaces=false,
    language=maple,
    backgroundcolor=\color{maplebgcolor},
    flexiblecolumns=true
}


\title{P4 Maple routines:\\blowup}
\author{}
\date{}

\setlength\marginparwidth{0cm}
\setlength\marginparsep{0cm}
\setlength\oddsidemargin{2cm}
\setlength\textwidth{\paperwidth}
\addtolength\textwidth{-2\oddsidemargin}
\addtolength\oddsidemargin{-1in}

\begin{document}
\maketitle

\section{Overview}

This part of the program implements the blow up procedure of a single singularity, and a local
study of all singularities encountered.  This may induce other blowups, and this is handled
in a recursive manner.

When reading the program diagonally, one observes a number of blow up routines: blow\_upx, blow\_upy,
blowing\_up2 and blowing\_up.  The last one is the entry point of this library.  The first one, blow\_upx
is a small routine to calculate the pullback vector field and to desingularize (by means of given
quasihomogeneous weights), in the positive or negative $x$-direction.  The similar task in the $y$-direction
is \emph{done by the same routine}, after applying a linear transformation.  The routine blow\_upy has
a completely different task: it performs the blow up and it analyses the singularities alltogether.

\section{Implementation}

Let us first load some libraries that are used throughout the program.

\begin{lstlisting}[name=blowup]
restart;
read( "tools.m" );
read( "writelog.m" );
read( "realroot.m" );
\end{lstlisting}

\subsection{Calculating weights: Newton diagram}

The first procedure calculates a list of points associated to the given vector field. During this procedure,
floating-point calculation is used (because not the coefficients, but the degrees are important).  For each
monomial $\alpha_{ij}x^iy^j$ in $\dot{x}$, a point $(i-1,j)$ is associated, and for each monomial
$\beta_{ij}x^iy^j$ in $\dot{y}$, a point $(i,j-1)$ is associated. The routine returns an unsorted list of points,
where double occurences are removed.

\begin{lstlisting}[name=blowup]
assoc_points := proc( vf )
    local Lf, Lg, expr, w;

    Lf := coeff_degree2( evalf(vf[1]), x, y );
    Lf := map( expr -> [ expr[1]-1, expr[2] ], Lf );

    Lg := coeff_degree2( evalf(vf[2]), x, y );
    Lg := map( expr -> [ expr[1], expr[2]-1 ], Lg );

    convert({op(Lf),op(Lg)},list);
end proc:
\end{lstlisting}

The next routine does the same, but removes a number of points that are irrelevant to describe the
newton diagram: it calculates the convex hull in some sense.  The resulting points are returned
as an unordered set.
\medskip

\noindent More in detail it does the following:  for each point $(x_1,y_1)$, it deletes all points $(x,y)$
for which both $x\geq x_1$ and $y\geq y_1$.  Then, for each pair of points $(x_1,y_1)$ and
$(x_2,y_2)$, it calculates the line through both points.  All other points $(x,y)$ with an $x$-coordinate
between $x_1$ and $x_2$ that lie above the line are deleted.

\begin{lstlisting}[name=blowup]
newton_diagram := proc( vf )
    local points, pk, pj, pL, k, j, L, hset, N, w, p;

    hset := assoc_points( vf );
    points := hset;

    N := nops(points);

    for k from 1 to N-1 do
        pk := points[k];
        if member( pk, hset ) then
            for j from k+1 to N do
                pj := points[j];
                if member( pj, hset ) then
                    if pk[1] <= pj[1] and pk[2] <= pj[2] then
                       hset := remove( p -> p=pj, hset );
                    else
                       for L from 1 to N do
                           pL := points[L];
                           if member( pL, hset ) and L <> k and L <> j then
                               if (pk[1] <= pL[1] and pk[2] <= pL[2]) or
                                  (pj[1] <= pL[1] and pj[2] <= pL[2]) then
                                     hset := remove( p -> p=pL, hset );
                               else
                                  if (pk[1] < pL[1] and pL[1] < pj[1]) or
                                     (pj[1] < pL[1] and pL[1] < pk[1]) then
                                       w := (pj[2]-pk[2])/(pj[1]-pk[1]);
                                       if w * (pL[1]-pj[1]) + pj[2] <= pL[2] then
                                           hset := remove( p -> p=pL, hset );
                                       end if
                                  end if
                               end if
                           end if
                       end do
                    end if
                end if
            end do
        end if
    end do;
    hset
end proc:
\end{lstlisting}

A third routine calculates the most important line from the Newton diagram.  It is
called with the list of points returned by \verb+newton_diagram+, and selects the
two points that define the most important line.

\begin{lstlisting}[name=blowup]
GAMMA_1 := proc( s )
    local L, p, q, r, k, sk, G;

    L := nops(s);
    p := s[1];
    q := s[2];

    if p[1] >= q[1] then
       p := s[2];
       q := s[1];
    end if;

    if L = 2 then
       G := [p,q]
    else
       r := s[3];
       if r[1] < p[1] then
          r := q;
          q := p;
          p := s[3]
       else
          if r[1] < q[1] then
             r := q;
             q := s[3]
          end if
       end if;

       for k from 4 to L do
          sk := s[k];
          if sk[1] < p[1] then
             r := q;
             q := p;
             p := sk
          else
             if sk[1] < q[1] then
                 r := q;
                 q := sk
             else
                 if sk[1] < r[1] then
                     r := sk
                 end if
             end if
          end if
       end do;
       if q[1] > 0 then
          G := [p,q]
       else
          G := [q,r]
       end if;
    end if;
    G
end proc:
\end{lstlisting}

Finally, the quasihomogeneous degree is obtained as a result from the previous routine.
This routine returns a list of three numbers, that can be used during the blow up:
The routine is called with a set of two points $\{s_1,s_2\}$, defining a line.  The return
value is a list of three integer values: the quasi-homogeneous degree for $x$, the degree for $y$,
and the desingularization index $d$ (to number of times to divide the vector field by the radius when
blowing up).

\begin{lstlisting}[name=blowup]
quasi_degree := proc( s )
   local h,alpha,beta,d;

   h := - (op(2,s[2]) - op(2,s[1])) / (op(1,s[2]) - op(1,s[1]));
   alpha := numer(h);
   beta := denom(h);
   d := alpha*op(1,s[1]) + beta*op(2,s[1]);
   [ alpha, beta, d ];
end proc:
\end{lstlisting}

\subsection{Some tools used during the blow up analysis}

The following routine determines the type of a center at $x=0$.  Given the reduced vector field $\dot{x}=f(x)$
with $f(0)=0$.  It is used when blowing up in the $x$-direction:

\begin{lstlisting}[name=blowup]
center_type1 := proc(f, x, L, dir )
    local s, g, m, b;

    g := first_term( f, x );
    m := max(0,degree( g, x ));
    b := sign( evalf(subs(x=1,g)) );
    if type(m, even) then
        s := [b,1,sign(evalf(L))]
    else
        s := [b,-1,sign(evalf(L))]
    end if;

    if dir=1 then
        if s = [-1,1,1] then return(10) end if;
        if s = [1,1,1] then return(9) end if;
        if s = [1,-1,1] then return(3) end if;
        if s = [-1,-1,1] then return(1) end if;
        if s = [-1,1,-1] then return(12) end if;
        if s = [1,1,-1] then return(11) end if;
        if s = [1,-1,-1] then return(2) end if;
        return(4)
    else
        if s = [-1,1,1] then return(9) end if;
        if s = [1,1,1] then return(10) end if;
        if s = [1,-1,1] then return(3) end if;
        if s = [-1,-1,1] then return(1) end if;
        if s = [-1,1,-1] then return(11) end if;
        if s = [1,1,-1] then return(12) end if;
        if s = [1,-1,-1] then return(2) end if;
        return(4)
    end if;
end proc:
\end{lstlisting}


A variant is used when blowing up in the $y$-direction.  In fact, while analysing this routine, it
seems to me that one could simply call {\slshape center\_type1(f,x,L,-dir)}.

\begin{lstlisting}[name=blowup]
center_type2 := proc(f, x, L, dir )
    local s, g, m, b;

    g := first_term( f, x );
    m := max(0,degree( g, x ));
    b := sign( evalf(subs(x=1,g)) );
    if type(m, even) then
        s := [b,1,sign(evalf(L))]
    else
        s := [b,-1,sign(evalf(L))]
    end if;

    if dir=1 then
        if s = [-1,1,1] then return(9) end if;
        if s = [1,1,1] then return(10) end if;
        if s = [1,-1,1] then return(3) end if;
        if s = [-1,-1,1] then return(1) end if;
        if s = [-1,1,-1] then return(11) end if;
        if s = [1,1,-1] then return(12) end if;
        if s = [1,-1,-1] then return(2) end if;
        return(4)
    else
        if s = [-1,1,1] then return(10) end if;
        if s = [1,1,1] then return(9) end if;
        if s = [1,-1,1] then return(3) end if;
        if s = [-1,-1,1] then return(1) end if;
        if s = [-1,1,-1] then return(12) end if;
        if s = [1,1,-1] then return(11) end if;
        if s = [1,-1,-1] then return(2) end if;
        return(4)
    end if;
end proc:
\end{lstlisting}

The following code implements a small routine calculating $x^k$, but avoiding the $0^0$ error by defining
$0^0 = 1$.  This is useful when evaluating power series or polynomials.

\begin{lstlisting}[name=blowup]
power := proc(a,b)
    if b = 0 then 1 else a^b end if;
end proc:
\end{lstlisting}

The next two routines yield sorted list of points in the plane.  The points are sorted with respect to their
$x$-coordinates, either in increasing order or decreasing order.  The routines are used to sort the singularities
in the blow up charts in a consistent way so that after blow down the separatrices are ordered.

\begin{lstlisting}[name=blowup]
sort_set_pos := proc( s, y0, r )
    local a, k, L;

    a := [$1..nops(s)+1];
    k := 1;
    for L from 1 to nops(s) do
        if reduce_llt( y0, op(1, s[L]) ) and k=L then
            a[k] := [y0,r];
            k := k+1
        end if;

        a[k] := s[L];
        k := k+1;
    end do;
    if k=L then
        a[k] := [y0,r];
    end if;
    a;
end proc:
\end{lstlisting}

\begin{lstlisting}[name=blowup]
sort_set_neg := proc( s, y0, r )
    local a, k, L;

    a := [$1..nops(s)+1];
    k := 1;
    for L from 1 to nops(s) do
        if reduce_gt( y0, op(1, s[L]) ) and k=L then
            a[k] := [y0,r];
            k := k+1
        end if;

        a[k] := s[L];
        k := k+1;
    end do;
    if k=L then
        a[k] := [y0,r];
    end if;
    a;
end proc:
\end{lstlisting}

\subsection{Separatrice routines}

Throughout the blow up process, some ``global'' variables are kept.  There, all separatrices are stored,
and the history of coordinate changes is kept:
\begin{itemize}
\item {\sl NumXPos}: number of separatrices found in the positive $x$-direction.
\item {\sl ManXPos}: coordinate changes during the blow up in the positive $x$-dir.
\item {\sl NumXNeg}: number of separatrices found in the negative $x$-direction.
\item {\sl ManXNeg}: coordinate changes during the blow up in the negative $x$-dir.
\item {\sl NumberY}: either 0 or 1 singularity studied in the $y$-direction.
\item {\sl ManYPos}: coordinate changes during the blow up in the positive $y$-dir.
\item {\sl ManYNeg}: coordinate changes during the blow up in the negative $y$-dir.
\end{itemize}

The number of separatrices in the $y$-directions is either $0$ or $2$ ($1$ in the positive direction, and
$1$ in the negative direction).  One can indeed show that with a quasi-homogeneous blow up, if there is
a singularity in the positive $y$-direction (in the origin), then it is automatically nondegenerate, hence
no further blowups are necessary.
\smallskip

Each separatrice is described by a list of 7 items: [ hist, point, trans, vf, sep, type, roundedflag ].
Let us describe this structure:

\begin{itemize}
\item   `hist': is a history list of blow up transformations, that are given in an ordered list.
        Each item of the history list is a blow up transformation, represented by $(a_1,\dots,a_8)$ of the form
        \[
            (x_0,y_0)\to (x_1,y_1)\colon\quad
            x_1 = a_1 + a_3x_0^{a_5}y_0^{a_6},\quad
            y_1 = a_2 + a_4x_0^{a_7}y_0^{a_8}
        \]
\item
        `trans': is a linear 2x2 matrix bringing the linear part of the vector field in normal form.  It is
        a transformation of the form
\item
        `point': two-coordinate representation of the singular point.  These are coordinates w.r.t.~the blow up
        space, i.e.~in the above form they are $(x_2,y_2)$-coordinates.
\item
        `vf': vector field $[\dot{x},\dot{y}]$ in which coordinates????
\item
        `sep': separatrix in blow up coordinates (i.e.~in $(x_2,y_2)$-coordinates).
\end{itemize}

During the blow up process, whenever a separatrice is found (while studying in the $x$-directions),
it is stored in a table, using the following routine:

\begin{lstlisting}[name=blowup]
put_separatrice := proc( xpos, stype, transformations, pt, trans, vec_field, sep )
    local newentry;
    global NumXPos, NumXNeg, ManXPos, ManXNeg;

    newentry := [transformations,pt,trans,vec_field,sep,stype,rounded];

    if xpos=1 then
        NumXPos := NumXPos + 1;
        ManXPos := [ op(ManXPos), newentry ];
    else
        NumXNeg := NumXNeg + 1;
        ManXNeg := [ op(ManXNeg), newentry ];
    end if;
end proc:
\end{lstlisting}

The following routine is important, because it recalculates a separatrice in another coordinate system (it
actually performs the blow down of a separatrice).
Notice, when reading the routine that $sep[3]$ is a transformation matrix and $sep[1]$ is a history
list of transformations, to be processed "first-in-last-out".  Each element of the history list is an $8$-item list
$(a_1,\dots a_8)$ defining the transformation
\[
    (x_0,y_0)\to (x_1,y_1)\colon\quad
    x_1 = a_1 + a_3x_0^{a_5}y_0^{a_6},\quad
    y_1 = a_2 + a_4x_0^{a_7}y_0^{a_8}
\]
The separatrix was originally stored as $sep[5]$, and is an expression in $t$.  The first transformation is
given by
\[
    \left(\begin{array}{c}x_0\\y_0\end{array}\right) =
    \left(\begin{array}{c}pt[1]\\pt[2]\end{array}\right) + \left(\begin{array}{cc} \hbox{trans}[1] & \hbox{trans}[2]    \\
    \hbox{trans}[3] & \hbox{trans}[4]\end{array}\right)
    \left(\begin{array}{c}t\\sep[5]\end{array}\right)
\]

\begin{lstlisting}[name=blowup]
make_transformation := proc(sep)
    local L, hist, pt, trans, s, x0, y0, x1 ,y1, a;

    hist := sep[1];
    pt := sep[2];
    trans := sep[3];
    s := sep[5];

    x0 := pt[1] + trans[1]*t + trans[2]*s;
    y0 := pt[2] + trans[3]*t + trans[4]*s;

    L := nops(hist);
    while L > 0 do
        a := hist[L];
        x1 := a[1] + a[3]*power(x0,a[5])*power(y0,a[6]);
        y1 := a[2] + a[4]*power(x0,a[7])*power(y0,a[8]);
        x0 := optimizepolynomial1(x1,t);
        y0 := optimizepolynomial1(y1,t);
        if (x0=0) and (y0=0) then
            L := 0
        else
            L := L-1
        end if
    end do;

    [x0,y0];
end proc:
\end{lstlisting}


The next routine is called at the end of the entire blow up routine, to collect all separatrices in a list,
and for each separatrix to calculate it in original blow down coordinates.

\begin{lstlisting}[name=blowup]
append_trajectors := proc()
    local man1, man2, j;

    print( "ManYNeg = ", ManYNeg );
    print( "ManYPos = ", ManYPos );
    print( "ManXNeg = ", ManXNeg );
    print( "ManXPos = ", ManXPos );

    man1:= [];
    if NumberY <> 0 then
        man2 := make_transformation( ManYNeg );
        man1 := [op(man1), [op(ManYNeg),man2]];
    end if;

    for j from 1 to NumXPos do
        man2 := make_transformation( ManXPos[j] );
        man1 := [op(man1), [op(ManXPos[j]),man2]];
    end do;

    if NumberY <> 0 then
        man2 := make_transformation( ManYPos );
        man1 := [op(man1), [op(ManYPos),man2]];
    end if;

    for j from 1 to NumXNeg do
        man2 := make_transformation( ManXNeg[j] );
        man1 := [op(man1), [op(ManXNeg[j]),man2]];
    end do;

    man1
end proc:
\end{lstlisting}

After calling this routine, each separatrice item is a list of \emph{8 items}.  Indeed, for
each trajectory, the above procedure has added an item, which was the result of make\_transformation.
\medskip

After collecting all separatrices, it is possible that this function is called (in case of a point at
infinity):

\begin{lstlisting}[name=blowup]
split_trajectors := proc( sep )
    local sep_pos, sep_neg, j, s, g;
    sep_pos := [];
    sep_neg := [];
    for j from 1 to nops(sep) do
        s := op( 8, sep[j]);
        if s[2] <> 0 then
            g := sign( evalf( subs( t=1, first_term( diff(s[2],t), t ) ) ) );
            if g > 0 then
                sep_pos := [ op(sep_pos), sep[j] ];
            else
                sep_neg := [ op(sep_neg), sep[j] ];
            end if;
        end if;
    end do;
    [sep_pos, sep_neg];
end proc:
\end{lstlisting}
The split routine analysis a list of separatrices, and divides the separatrices in two classes:
positive and negative separatrices.  In each separatrix definition $sep[j]$, it extracts
the last element $s$, which is of the form $[f(t),g(t)]$.  It analysis the sign of $g'(t)$, to make
the distinction between separatrices in the upper half or lower half plane.
\medskip

The next routine gives information on the terminal, and in the result file about
all the trajectories that are found.  The trajectories are given in an ordered
list, and between each two trajectories, the routine determines the type of sector
that is found (elliptic, hyperbolic or parabolic).  Finally, the index is calculated.
There is no return value.

\begin{lstlisting}[name=blowup]
write_trajectors := proc( sep, x0, y0 )
    local L, w1, s1, s2, j, el, hyp, w2, dl1, dl2, dl3, dl4, dl5;

    L := nops(sep);
    el := 0;
    hyp := 0;
    dl1 := "";
    dl2 := " ";
    dl3 := "";
    dl4 := "";
    dl5 := "";

    if L <> 0 then
        openfile( result_file, terminal );
        writef( "\n\nThe invariant separatrices are: (t>0)\n" );
        openfile( terminal );

        w1 := {1,3,5,7,9,10,13,14};  # unstable ones
        w2 := {2,4,6,8,11,12,15,16}; # stable ones
        s1 := op(6,sep[1]);
        writef( sprintf( " [%d]" ,s1) );
        if member( s1, w1 ) then
            openfile( result_file, terminal );
            writef( "[01]: %a : unstable\n", op(8,sep[1]) );
            openfile( terminal );
            dl1 := cat(dl1, "   |    ");
            dl2 := cat(dl2,  "  ^  ");
            dl3 := cat(dl3, "   |    ");
        else
            openfile( result_file, terminal );
            writef( "[01]: %a : stable\n", op(8,sep[1]) );
            openfile( terminal );
            dl1 := cat(dl1, "   |    ");
            dl2 := cat(dl2,  "  v  ");
            dl3 := cat(dl3, "   |    ");
        end if;
        if member( s1, {2,3,5,6,10,12,13} ) then
            dl4 := cat(dl4 ,  "-<-*" );
        else
            if member( s1, {1,4,7,8,9,11,15} ) then
            dl4 := cat( dl4,  "->-*" );
            else
                dl4 := cat(dl4 , "****" );
            end if;
        end if;
        if member( s1, {1,4,7,8,10,12,16} ) then
            dl4 := cat(dl4, "-<- " );
        else
            if member( s1, {2,3,5,6,9,11,14} ) then
            dl4 := cat(dl4, "->- " );
            else
                dl4 := cat(dl4, "*** " );
            end if;
        end if;
        dl5 := cat( dl5, sprintf( "  %02d    ", 1 ) );

        for j from 2 to L do
            s2 := op(6, sep[j]);
            if (member( s1,{1,7,10}) and member( s2, {2,6,12} )) or
               (member( s1,{2,6,11}) and member( s2, {1,7,9} )) then
                  openfile( result_file, terminal );
                  writef( "hyperbolic sector\n" );
                  openfile( terminal );
                  hyp := hyp + 1;
                  dl2 := cat(dl2, "HYP");
            else
                if (member(s1,{3,5,9,14}) and member(s2,{4,8,11,15})) or
                   (member(s1,{4,8,12,16}) and member(s2,{3,5,10,13})) then
                    openfile( result_file, terminal );
                    writef( "elliptic sector\n" );
                    openfile( terminal );
                    el := el + 1;
                    dl2 := cat(dl2, "ELL");
                else
                    openfile( result_file, terminal );
                    writef( "parabolic sector\n" );
                    openfile( terminal );
                    dl2 := cat(dl2, "PAR");
                end if
            end if;

            s1 := s2;
            writef( sprintf( " [%d]" ,s1) );

            if member( s1, w1 ) then
                openfile( result_file, terminal );
                writef( "[%02d]: %a : unstable\n", j, op(8,sep[j]) );
                openfile( terminal );
                dl1 := cat(dl1, "   |    ");
                dl2 := cat(dl2,  "  ^  ");
                dl3 := cat(dl3, "   |    ");
            else
                openfile( result_file, terminal );
                writef( "[%02d]: %a : stable\n", j, op(8,sep[j]) );
                openfile( terminal );
                dl1 := cat(dl1, "   |    ");
                dl2 := cat(dl2,  "  v  ");
                dl3 := cat(dl3, "   |    ");
            end if;
            if member( s1, {2,3,5,6,10,12,13} ) then
                dl4 := cat(dl4 ,  "-<-*" );
            else
                if member( s1, {1,4,7,8,9,11,15} ) then
                dl4 := cat( dl4,  "->-*" );
                else
                    dl4 := cat(dl4 , "****" );
                end if;
            end if;
            if member( s1, {1,4,7,8,10,12,16} ) then
                dl4 := cat(dl4, "-<- " );
            else
                if member( s1, {2,3,5,6,9,11,14} ) then
                dl4 := cat(dl4, "->- " );
                else
                    dl4 := cat(dl4, "*** " );
                end if;
            end if;
            dl5 := cat( dl5, sprintf( "  %02d    ", j ) );
        end do;

        s1 := op(6, sep[L]);
        s2 := op(6, sep[1]);

        if (member( s1,{1,7,10}) and member( s2, {2,6,12} )) or
            (member( s1,{2,6,11}) and member( s2, {1,7,9} )) then
                openfile( result_file, terminal );
                writef( "hyperbolic sector\n" );
                openfile( terminal );
                hyp := hyp + 1;
                dl2 := cat(dl2, "HYP");
        else
            if (member(s1,{3,5,9}) and member(s2,{4,8,11})) or
                (member(s1,{4,8,12}) and member(s2,{3,5,10})) then
                openfile( result_file, terminal );
                writef( "elliptic sector\n" );
                openfile( terminal );
                el := el + 1;
                dl2 := cat(dl2 , "ELL");
            else
                openfile( result_file, terminal );
                writef( "parabolic sector\n" );
                openfile( terminal );
                dl2 := cat(dl2, "PAR");
            end if
        end if
    end if;

    openfile( result_file, terminal );
    writef( "The singularity (%a,%a) has index %a\n", x0, y0, 1+(el-hyp)/2 );
    writef( "Sector representation (counterclockwise):\n" );
    writef( dl1 ); writef( "\n" );
    writef( dl2 ); writef( "\n" );
    writef( dl3 ); writef( "\n" );
    writef( dl4 ); writef( "\n" );
    writef( dl5 ); writef( "\n\n" );
    openfile( terminal );
end proc:
\end{lstlisting}

\subsection{Manifold business}

Given a vector field, for which the linear part is in Jordan normal form, we can calculate formally the invariant
separatrices.  For semihyperbolic vector fields, this means we can calculate the center separatrice
(\verb+center_manifold_blow_up+), and the hyperbolic separatrice (\verb+manifold_blow_up+).  For hyperbolic
vector fields, we can do similar calculations, but have to cut off the series to avoid possible division by zero
in case of resonances between the eigenvalues. This is done in \verb+manifold2_blow_up+.
\smallskip

It seems that separatrice testing is not done for the separatrices that are calculates in these routines.

\begin{lstlisting}[name=blowup]
manifold_blow_up := proc(_g,n)
    local lambda1, lambda2, g, f, a, dfx, k, h;

    g := optimizevf(_g,x,y);
    lambda1 := coeff( coeff( g[1], x, 1 ), y, 0 );
    lambda2 := coeff( coeff( g[2], y, 1 ), x, 0 );

    f := add( a[k]*x^k, k=2..n );
    dfx := add( k*a[k]*x^(k-1), k=2..n );

    h := expand( subs( y=f, g[2] - lambda2*y ) - dfx * subs( y=f, g[1]-lambda1*x ) );

    for k from 2 to n do
        h := eval(h, a[k] = coeff(h,x,k)/(lambda1*k - lambda2) );
        f := eval(f,a[k] = coeff(h,x,k)/(lambda1*k - lambda2) );
    end do;

    return subs(x=t,f);
end proc:

manifold2_blow_up := proc(g,n)
    local L1,L2,f,dfx,k,i,h,a,n1;

    L1:=coeff(coeff(g[1],x,1),y,0);
    L2:=coeff(coeff(g[2],y,1),x,0);
    f := add( a[k]*x^k, k=2..n );
    dfx := add( k*a[k]*x^(k-1), k=2..n );
    h := subs(y=f, g[2] - L2*y) - dfx*subs(y=f, g[1] - L1*x);
    k := 2:
    n1 := n;
    while k < n+1 do
        if reduce_nneq( k*L1-L2,0) then
            h :=  eval( h, a[k] = coeff(h,x,k)/(k*L1-L2) );
            f :=  eval( f, a[k] = coeff(h,x,k)/(k*L1-L2) );
            k := k+1;
        else
            n1 := k-1;
            k := n+1;
        end if;
    end do;
    add( coeff(f,x,i)*t^i, i=2..n1);
end proc:

center_manifold_blow_up := proc( g, n )
    local vf1, vf2, h1, h2, k, L2, sep, _y, lambda, ok, n1, dx, b, m, am;

    if rounded then vf1 := evalf(g[1]); vf2 := evalf(g[2]); else vf1 := g[1]; vf2 := g[2]; fi;
    L2 := coeff(collect(subs(x=0,g[2]),y),y,1);
    sep := 0; n1 := n;
    for k from 2 to n1 do
        lambda := - coeff(collect( subs(y=0,vf2),x),x,k) / L2;
        if not rounded and user_simplify then lambda := user_simplifycmd(lambda) fi;
        h1 := expand( subs( y=_y+lambda*x^k,vf1) );
        h2 := expand( subs( y=_y+lambda*x^k,vf2) - lambda*k*x^(k-1)*h1);
        vf1 := subs( _y=y, h1 );
        vf2 := subs( _y=y, h2 );
        sep := sep + lambda*x^k;
        dx := optimizepolynomial1( subs( y=0, vf1 ), x ); if dx = 0 then n1 := n1 + 1 fi
    od;
     dx := optimizepolynomial1( subs( y=0, vf1 ), x );
     [ sep, dx, first_term(dx,x) ];
end:
\end{lstlisting}

\subsection{The blow up routine(s)}

The following routine is called from outside the blowup module, to blow up a single singularity
and examine its type.  This is hence the entry point, and calls subsequently the other routines inside
this module.  This routine itself is not called recursively.
\medskip

The separatrices are gathered in a list, and are returned.  For points at infinity, the behaviour
is slightly different: the separatrices are divided in two parts: separatrices in the upper halfplane,
and separatrices in the lower halfplane.  During the study at infinity, only one half is needed
to draw the phase portrait.  Except when there is a line of singularities at infinity: in that
cases, the line is desingularized and new, isolated, points at infinity are studied as if they were
finite points.  Both halfs of the blow up study are analyzed, because they are needed to study
the symmetric singularity at the other side of the circle of infinity.  For more information, look up the
study at infinity.

The main entry point is given below: the fourth argument is a boolean, that is true if the singularity
to be studied is in the finite region, or false if it is on the circle at infinity.

\begin{lstlisting}[name=blowup]
ManXPos := [];
ManYPos := [];
ManXNeg := [];
ManYNeg := [];
NumXPos := 0;
NumXNeg := 0;
NumberY := 0;

blowing_up := proc( vf, x0, y0, finite_point )
    local r, vf2, n_d, g1, q_degree, z, trajectors;
    global ManXPos, ManXNeg, ManYPos, ManYNeg,
           NumXPos, NumXNeg, NumberY;

    if save_all then openfile( result_file, terminal ) end if;
    writef( "blowing up %a in point (%a,%a).\n", vf, x0, y0 );
    openfile( terminal );

    if x0 <> 0 or y0 <> 0 then
        vf2 := optimizevf( translation( vf, x, y, x0, y0 ), x, y );
        if save_all then openfile( result_file, terminal ) end if;
        writef( "moving (%a,%a) to the origin.\n", x0, y0 );
        writef( "new vector field: %a\n", vf2 );
        openfile( terminal );
    else
        vf2 := optimizevf( vf, x, y );
    end if;

    #calculate newton_diagram (in symbolic mode)

    n_d := newton_diagram( vf2 );
    g1 := GAMMA_1( n_d );
    q_degree := quasi_degree( g1 );

    if save_all then openfile( result_file, terminal ) end if;
    writef( "Newton diagram: %a\n", n_d );
    writef( "g1 = %a\n", g1 );
    writef( "[alpha,beta,degree] = %a\n", q_degree );
    writef( "\n" );
    writef( "1. Blowing up in the positive x-direction\n" );
    writef( "-----------------------------------------\n" );
    writef( "\n" );
    writef( "blowing up level 1\n" );
    writef( "------------------\n" );
    writef( "blowing up %a in point (0,0)\n", vf2 );
    openfile( terminal );

    z:=blow_upx(vf2,q_degree,1,x,y);
    if save_all then openfile( result_file, terminal ) end if;
    writef( "blowing up vectorfield %a\n", z[1] );
    openfile( terminal );

    sing_eigenv(z, 1, 1, [[ x0, y0, 1, 1, q_degree[1], 0,
                    q_degree[2], 1, q_degree[3] ]], 1 );

    if save_all then openfile( result_file, terminal ) end if;
    writef( "---------------------------------------------------------\n" );
    writef( "\n" );
    writef( "2. Blowing up in the negative x-direction\n" );
    writef( "-----------------------------------------\n" );
    writef( "\n" );
    writef( "\n" );
    writef( "blowing up level 1\n" );
    writef( "------------------\n" );
    writef( "blowing up %a in point (0,0)\n", vf2 );
    openfile( terminal );

    z:=blow_upx(vf2,q_degree,-1,x,y);

    if save_all then openfile( result_file, terminal ) end if;
    writef( "blowing up vectorfield %a\n", z[1] );
    openfile( terminal );

    sing_eigenv(z, -1, 1, [[ x0, y0, -1, 1, q_degree[1], 0,
                q_degree[2], 1, q_degree[3] ]], -1 );

    # Is (0,pi/2) a singular point? LINE OF SING --> SEMI-EL POINT IS ALWAYS NORMALLY HYPERBOLIC, SO NO CONTACT POINT!

    if op( 1, g1[1] ) > -1 and z[3] <> 0 then
        if save_all then openfile( result_file, terminal ) end if;
        writef( "3. Blowing up in the positive y-direction\n" );
        writef( "-----------------------------------------\n" );
        openfile( terminal );

        ManYPos := blow_upy( vf2, q_degree, 1, [ [ x0,y0,1,1,1,
                    q_degree[1],0,q_degree[2],q_degree[3] ]] );

        NumberY := 1;

        if save_all then openfile( result_file, terminal ) end if;
        writef( "\n" );
        writef( "4. Blowing up in the negative y-direction\n" );
        writef( "-----------------------------------------\n" );
        openfile( terminal );

        ManYNeg := blow_upy( vf2, q_degree, -1, [ [ x0,y0,1,-1,1,
                    q_degree[1],0,q_degree[2],q_degree[3]]] );
    end if;

    trajectors:=append_trajectors();
    write_trajectors( trajectors, x0, y0 );
    if not(finite_point) then
        trajectors:=split_trajectors(trajectors)
    end if;

    NumXPos := 0;
    NumXNeg := 0;
    NumberY := 0;
    ManXPos := [];
    ManXNeg := [];
    ManYPos := [];
    ManYNeg := [];

    trajectors;
end proc:
\end{lstlisting}

The next routine calculates the pull back of a vector field under the blow up map in the $x$-direction.
There is no analogon for calculating the pull back of a vector field under the blow up map in the $y$-direction:
this latter task is simply done by combining \verb+blow_upx+ with linear transformations.

\begin{lstlisting}[name=blowup]
blow_upx := proc( vf, q, dir, x, y )
    local alpha, beta, d, vx, vy, pg, qg, tg, f, g, r, s, delta;

    alpha := q[1];
    beta := q[2];
    d := q[3];
    vx := 0;
    vy := 0;
    pg := 0;
    qg := 0;
    tg := 0;

    f := optimizepolynomial2( vf[1], x, y );
    while f <> 0 do
        g := nlterm( f, x, y );
        r := max(0,degree(g,x))-1;
        s := max(0,degree(g,y));
        delta := alpha*r + beta*s;
        vx := vx + dir * subs(x=dir,g)*x^(delta+1-d);
        vy := vy - dir * x^(delta-d) * beta*y*subs(x=dir,g);
        if delta = d then
            pg := pg + dir*subs(x=dir,g);
            tg := tg - dir*beta*y*subs(x=dir,g);
        end if;
        f := optimizepolynomial2( f-g, x, y );
    end do;

    f := optimizepolynomial2( vf[2], x, y );
    while f <> 0 do
        g := nlterm( f, x, y );
        r := max(0,degree(g,x));
        s := max(0,degree(g,y))-1;
        delta := alpha*r + beta*s;
        vy := vy + alpha*x^(delta-d)*subs(x=dir,g);
        if delta=d then
            qg := qg + subs(x=dir,g);
            tg := tg + alpha*subs(x=dir,g);
        end if;
        f := optimizepolynomial2( f-g, x, y );
    end do;

    [ [vx, vy], [pg,qg], tg ];
end proc:
\end{lstlisting}

The \verb+blow_upy+ routine has a different task.  Besides calculating the pull back (using
a combination of \verb+blow_upy+ and linear transformations), it determines the type of
the singularity at the origin $(0,0)$.  The routine is not called if this point is not singular.
One can show that the singularity is never degenerate, and hence always elementary or semi-elementary.
This means that no iterated blowups are necessary in the $y$-direction.

\begin{lstlisting}[name=blowup]

blow_upy := proc( vec_field, q_degree, direction, transformations )
    local z, vf2, lambda1, lambda2, vec, g, man, sep;

    vf2 := transformation( vec_field, 0, 1, 1, 0 );
    z := blow_upx( vf2, [ q_degree[2], q_degree[1], q_degree[3] ], direction, x, y );
    vf2 := transformation( z[1], 0, 1, 1, 0 );

    if save_all then openfile( result_file, terminal ) end if;
    writef( "blowing up vector field %a\n", vf2 );

    lambda1 := subs( y=0, op(1,z[2]) );
    lambda2 := subs( y=0, diff( z[3], y ) );

    writef( "(0,0) is a singular point with eigenvalues %a, %a.\n", lambda2, lambda1 );
    openfile( terminal );

    # which type of singularity ?

    if reduce_eeq(lambda1,0) or reduce_eeq(lambda2,0) or reduce_gteq(0,lambda1*lambda2) then
        vec := subs( x=0, y=0, diff( op(2, z[1]),x ) ) / (lambda1-lambda2);
        g := transformation( z[1], 1, 0, vec, 1 );

        if save_all then openfile( result_file, terminal ) end if;
        writef( "Transform this vector field so that the linear part is diagonal\n" );
        writef( "The new vector field is %a.\n", transformation(g, 0,1,1,0) );
        openfile( terminal );

        if reduce_gt(lambda1,0) then
            sep := manifold_blow_up( g, taylor_level );
            if save_all then openfile( result_file, terminal ) end if;
            writef( "unstable separatrice x=%a", subs(t=y,sep) );
            openfile( terminal );
            if reduce_nneq(lambda2,0) then
                man := [ transformations, [0,0], [vec,1,1,0], g, sep, 1, rounded ]
            else
                man := [ transformations, [0,0], [vec,1,1,0], g, sep,
                        center_type2(subs(y=0,vf2[1]),x,lambda1,direction), rounded ];
            end if
        else
            if reduce_gt( 0, lambda1 ) then
                sep := manifold_blow_up( g, taylor_level );
                if save_all then openfile( result_file, terminal ) end if;
                writef( "stable separatrice x=%a\n",subs(t=y,sep) );
                openfile( terminal );
                if reduce_nneq(lambda2,0) then
                    man := [ transformations, [0,0], [vec,1,1,0], g, sep, 2, rounded ]
                else
                    man := [ transformations, [0,0], [vec,1,1,0], g, sep,
                        center_type2( subs(y=0,vf2[1]),x,lambda1,direction), rounded ];
                end if
            else
                sep:=center_manifold_blow_up( g, taylor_level );
                if save_all then openfile( result_file, terminal ) end if;
                writef( "center manifold x=%a\n",subs(x=y,sep[1]) );
                writef( "on center manifold y'=%a\n",subs(x=y,sep[2]) );
                openfile( terminal );
                if reduce_gt(lambda2,0) then
                    if subs(x=1,sep[-1]) > 0  then
                        man := [ transformations, [0,0], [vec,1,1,0], g, subs(x=t,sep[1]), 5, rounded ]
                    else
                        man := [ transformations, [0,0], [vec,1,1,0], g, subs(x=t,sep[1]), 6, rounded ]
                    end if
                else
                    if subs(x=1,sep[-1]) > 0 then
                        man := [ transformations, [0,0], [vec,1,1,0], g, subs(x=t,sep[1]), 7, rounded ]
                    else
                        man := [ transformations, [0,0], [vec,1,1,0], g, subs(x=t,sep[1]), 8, rounded ]
                    end if
                end if
            end if
        end if
    else
        if reduce_gt(lambda1,0) then
            if reduce_nneq(lambda1,lambda2) then
                vec := subs(x=0,y=0,diff(op(2,z[1]),x))/(lambda1-lambda2);
                g := transformation( z[1], 1,0,vec,1 );
                sep := manifold2_blow_up( g, taylor_level );
                man := [ transformations, [0,0], [vec,1,1,0], g, sep, 3, rounded ];
            else
                man := [ transformations, [0,0], [0,1,1,0], z[1], 0, 3, rounded ]
            end if
        else
            if reduce_nneq(lambda1,lambda2) then
                vec := subs(x=0,y=0,diff( op(2,z[1]),x))/(lambda1-lambda2);
                g := transformation( z[1], 1,0,vec,1 );
                sep := manifold2_blow_up( g, taylor_level );
                man := [ transformations, [0,0], [vec,1,1,0], g, sep, 4, rounded ]
            else
                man := [ transformations, [0,0], [0,1,1,0], z[1], 0, 4, rounded ];
            end if
        end if;
    end if;
    return man;
end proc;

\end{lstlisting}

The procedure \verb+sing_eigenv+ has the following task: ????

\begin{lstlisting}[name=blowup]

sing_eigenv := proc( z, dir, lvl, transformations, xpos )
    local r,s,L,L1,L2,h,pt,vec,ff,trans,g,sep;

    if z[3] <> 0 then
        h := find_real_roots1( z[3], y );
        s := [];
        for L from 1 to nops(h[1]) do
            if xpos=1 then
                s := sort_set_pos( s, op(L, h[1]), false );
            else
                s := sort_set_neg( s, op(L, h[1]), false );
            end if;
        end do;
        for L from 1 to nops(h[2]) do
            if xpos=1 then
                s := sort_set_pos( s, op(L, h[2]), true );
            else
                s := sort_set_neg( s, op(L, h[2]), true );
            end if;
        end do;
        for L from 1 to nops(s) do
            pt := op(1, s[L]);
            L1 := eval( op(1,z[2]), y=pt );
            L2 := eval( diff( z[3], y ), y=pt );
            if op(2,s[L]) then
                L1 := evalf( L1 );
                L2 := evalf( L2 );
            end if;

            if save_all then openfile( result_file, terminal ) end if;

            if reduce_eeq( L1, 0 ) and reduce_eeq( L2, 0 ) then
                writef( "(0,%a) is a non-elementary point\n", pt );
                openfile( terminal );
                blowing_up2( z[1], pt, lvl+1, transformations, xpos, false );
            else
                if reduce_eeq( L1, 0 ) or reduce_eeq( L2, 0 ) or reduce_gteq( 0, L1*L2 ) then
                    vec := diff(op(2,z[1]),x);
                    vec := subs(x=0,y=pt,vec / (L1-L2) );
                    if op(2,s[L]) then
                        vec := evalf(vec);
                    end if;
                    ff := translation( z[1], x, y, 0, pt );
                    if op(2,s[L]) then
                        ff := evalf(ff);
                    end if;
                    trans := [1,0,vec,1];
                    g := transformation( ff, 1,0,vec,1 );
                    if op(2,s[L]) then
                        g := evalf(g);
                    end if;

                    writef( "(0,%a) is a singular point with eigenvalues %a,%a\n", pt, L1, L2 );
                    writef( "moving this point to the origin\n" );
                    writef( "the new vector field is %a\n", ff );
                    writef( "transform this vector field so that the linear part is diagonal\n" );
                    writef( "the new vector field is %a\n", g );

                    if reduce_gt( L1, 0 ) then
                        sep := manifold_blow_up( g, taylor_level );
                        writef( "unstable separatrice y=%a\n", subs(t=x,sep) );

                        if reduce_nneq( L2, 0 ) then
                            put_separatrice( xpos, 1, transformations, [0,pt], trans, g, sep );
                        else
                            put_separatrice( xpos, center_type1( subs(x=0, g[2]), y, L1, dir ),
                                            transformations, [0,pt], trans, g, sep );
                        end if
                    else
                        if reduce_gt( 0, L1 ) then
                            sep := manifold_blow_up( g, taylor_level );
                            writef( "stable separatrice y=%a\n", subs(t=x,sep) );
                            if reduce_nneq( L2, 0 ) then
                                put_separatrice( xpos, 2, transformations, [0,pt], trans, g, sep );
                            else
                                put_separatrice( xpos, center_type1( subs(x=0,g[2]),y,L1,dir),
                                                transformations, [0,pt], trans, g, sep );
                            end if
                        else
                            sep := center_manifold_blow_up( g, taylor_level );
                            writef( "center manifold y=%a\n",sep[1] );
                            writef( "on center manifold is x'=%a\n", sep[2] );
                            if reduce_gt( L2, 0 ) then
                                if sign( evalf(subs(x=1,sep[3])) ) > 0 then
                                    put_separatrice( xpos, 5, transformations, [0,pt], trans, g, subs(x=t,sep[1]) )
                                else
                                    put_separatrice( xpos, 6, transformations, [0,pt], trans, g, subs(x=t,sep[1]) )
                                end if
                            else
                                if sign( evalf(subs(x=1,sep[3])) ) > 0 then
                                    put_separatrice( xpos, 7, transformations, [0,pt], trans, g, subs(x=t,sep[1]) )
                                else
                                    put_separatrice( xpos, 8, transformations, [0,pt], trans, g, subs(x=t,sep[1]) )
                                end if
                            end if
                        end if
                    end if
                else
                    if reduce_gt( L1, 0 ) then
                        writef( "(0,%a) is an unstable node with eigenvalues %a,%a\n", pt, L1, L2 );

                        if reduce_nneq( L1, L2 ) then
                            vec := diff( op(2,z[1]), x ) / (L1-L2);
                            vec := subs(x=0,y=pt,vec);
                            if op(2, s[L]) then
                                vec := evalf(vec)
                            end if;
                            ff := translation( z[1], x, y, 0, pt );
                            if op(2, s[L]) then
                                ff := evalf(ff);
                            end if;
                            trans := [1,0,vec,1];
                            g := transformation( ff, 1,0,vec,1 );
                            sep := manifold2_blow_up( g, taylor_level );
                            put_separatrice( xpos, 3, transformations, [0,pt], trans, g, sep );
                        else
                            put_separatrice( xpos, 3, transformations, [0,pt], [1,0,0,1],
                                            translation(z[1],x,y,0,pt), 0 );
                        end if
                    else
                        writef( "(0,%a) is a stable node with eigenvalues %a,%a\n", pt, L1, L2 );

                        if reduce_nneq( L1, L2 ) then
                            vec := diff( op(2,z[1]), x ) / (L1-L2);
                            vec := subs(x=0,y=pt,vec);
                            if op(2, s[L]) then
                                vec := evalf(vec)
                            end if;
                            ff := translation( z[1], x, y, 0, pt );
                            if op(2, s[L]) then
                                ff := evalf(ff);
                            end if;
                            trans := [1,0,vec,1];
                            g := transformation( ff, 1,0,vec,1 );
                            sep := manifold2_blow_up( g, taylor_level );
                            put_separatrice( xpos, 4, transformations, [0,pt], trans, g, sep );
                        else
                            put_separatrice( xpos, 4, transformations, [0,pt], [1,0,0,1],
                                            translation(z[1],x,y,0,pt), 0 );
                        end if
                    end if
                end if
            end if
        end do
    else
        #x=0 is a line of singularities !!!
        if save_all then openfile( result_file, terminal ) end if;

        writef( "x=0 is a line of singularities - point is star-like\n" );
        h := find_real_roots1( op(1,z[2]), y );
        s := [];
        for L from 1 to nops(h[1]) do
            if xpos=1 then
                s := sort_set_pos( s, op(L, h[1]), false );
            else
                s := sort_set_neg( s, op(L, h[1]), false );
            end if;
        end do;
        for L from 1 to nops(h[2]) do
            if xpos=1 then
                s := sort_set_pos( s, op(L, h[2]), true );
            else
                s := sort_set_neg( s, op(L, h[2]), true );
            end if;
        end do;
        writef( "the non-elementary singular points (contact points) are:\n" );
        for L from 1 to nops(h[1]) do
            writef( "(0,%a)\n", op(L, h[1]) );
        end do;
        for L from 1 to nops(h[2]) do
            writef( "(0,%a)\n", op(L, h[2]) );
        end do;
        openfile(terminal);
        for L from 1 to nops(s) do
            blowing_up2( z[1], op(1,s[L]), lvl+1, transformations, xpos, true );
        end do;
    end if;
    openfile(terminal);
end proc:
\end{lstlisting}

The next routine is used in the blow up procedure while blowing in the x-direction.
It makes a study at the corner of the blow up locus.  This study is very simple,
and cannot enhold further blowups: depending on the behaviour at the corner,
an extra separatrix is added or not.  If a separatrice is added, it is given by $\{y=x\}$ in
local coordinates. The routine does not return a value.  Observe that, in the corner
(intersection of two parts of the blow up locus), we already have two invariant separatrices,
but they are not important after blow down.

Comment: if $\dot{x}=x$ and $\dot{y}=y$, where $xy=0$ is the border of the blow-up locus near the corner, then the equality of the two eigenvalues
means that after blow down there is in fact saddle behaviour (dynamics points away from the origin).  The way the separatrix is defined, namely $y=x$ (or in the program lines $(x(t),y(t))=(t,t)$), is quite arbitrary and is not related to an actual orbit.

Comment 2: in the case of  a line of singularities we may have that ${x=0}$ is the line of singularities and that ${y=0}$ is the blow-up locus.  So the vector field along ${y=0}$ defines the slow-dynamics at the contact point.  We distinguish 4 cases: see tex file
%
%     |           |          |          |
%     |           |          |          |
%  -<-*****    ****->-    ->-****    ****-<-
%     13          14         15         16

\begin{lstlisting}[name=blowup]
blowing_upy2 := proc( vf, q_degree, dir, transformations, xpos, sing )
    local vf2, z, sx, sy;

    vf2 := transformation( vf, 0,1,1,0 );
    z := blow_upx( vf2, [ q_degree[2], q_degree[1], q_degree[3] ], dir, x, y );
    vf2 := transformation( z[1], 0,1,1,0 );

    if save_all then openfile(result_file,terminal) end if;
    if dir=1 then
        writef( "blowing up in the positive y-direction\n" );
    else
        writef( "blowing up in the negative y-direction\n" );
    end if;
    writef( "blowing up the vector field %a.\n", vf2 );
    openfile( terminal );

    if sing = false then
        sx := sign( evalf(subs( x=1, first_term( subs(y=0, vf2[1]), x ) ) ) );
        sy := sign( evalf(subs( y=1, first_term( subs(x=0, vf2[2]), y ) ) ) );

        if sx=sy then
            if sx>0 then
                put_separatrice( xpos, 3, transformations, [0,0], [1,0,0,1], vf2, t );
            else
                put_separatrice( xpos, 4, transformations, [0,0], [1,0,0,1], vf2, t );
            end if
        end if
    else
        if first_term( subs(y=0, vf2[1]), x ) = 0 then
            error( "case not treated: star-like points should become non star-like after second blowup.");
        end if;
        sx := sign( evalf(subs( x=1, first_term( subs(y=0, vf2[1]), x ) ) ) );

        if dir*xpos>0 then
            put_separatrice( xpos, 14-sx, transformations, [0,0], [1,0,0,1], vf2, t );
        else
            put_separatrice( xpos, 15-sx, transformations, [0,0], [1,0,0,1], vf2, t );
        end if
    end if
end proc:
\end{lstlisting}

Yet another variant:

\begin{lstlisting}[name=blowup]

blowing_up2 := proc( vf, pt, lvl, transformations, xpos, sing )
    local vf2, n_d, g1, q_degree, trf, z;

    if save_all then openfile( terminal, result_file ) end if;
    writef( "Blowing up level %d\n", lvl );
    writef( "------------------\n" );
    writef( "blowing up %a in point (0,%a)\n", vf, pt );

    if reduce_nneq( pt, 0 ) then
        vf2 := translation( vf, x, y, 0, pt );
        writef( "moving (0,%a) to the origin\n", pt );
        writef( "new vector field %a\n", vf2 );
    else
        vf2 := vf
    end if;

    n_d := newton_diagram( vf2 );
    g1 := GAMMA_1( n_d );
    q_degree := quasi_degree( g1 );

    writef( "Newton diagram %a\n", n_d );
    writef( "[alpha,beta,degree]=%a\n", q_degree );

    openfile( terminal );

    #first we blow up in the negative y-direction if xpos=1
    #else we blow up in the positive y-direction
    #we have to know what happens in the corner
    #only in the case that {X=0} is not a line of singularities

    if sing=false then
        if xpos=1 then
            trf := [0,pt,1,-1,1,q_degree[1],0,q_degree[2],q_degree[3]];
            blowing_upy2( vf2, q_degree, -1, [ op(transformations), trf ], xpos, false )
        else
            trf := [0,pt,1,1,1,q_degree[1],0,q_degree[2],q_degree[3]];
            blowing_upy2( vf2, q_degree, 1, [ op(transformations), trf ], xpos, false )
        end if
    else
        if xpos=1 then
            trf := [0,pt,1,-1,1,q_degree[1],0,q_degree[2],q_degree[3]];
            blowing_upy2( vf2, q_degree, -1, [ op(transformations), trf ], xpos, true )
        else
            trf := [0,pt,1,1,1,q_degree[1],0,q_degree[2],q_degree[3]];
            blowing_upy2( vf2, q_degree, 1, [ op(transformations), trf ], xpos, true )
        end if
    end if;
    if save_all then openfile( terminal, result_file ) end if;
    writef( "blowing up in the positive x direction\n" );
    z := blow_upx( vf2, q_degree, 1, x, y );
    writef( "blowing up vector field %a\n", z[1] );
    openfile( terminal );

    trf := [0,pt,1,1,q_degree[1],0,q_degree[2],1,q_degree[3]];
    sing_eigenv( z, 1, lvl, [ op(transformations), trf ], xpos );

    #now we blow up in the positive y direction if xpos=1
    #else we blow up in the negative y direction

    if sing=false then
        if xpos=1 then
            trf := [0,pt,1,1,1,q_degree[1],0,q_degree[2],q_degree[3]];
            blowing_upy2( vf2, q_degree, 1, [ op(transformations), trf ], xpos, false )
        else
            [0,pt,1,-1,1,q_degree[1],0,q_degree[2],q_degree[3]];
            blowing_upy2( vf2, q_degree, -1, [ op(transformations), trf ], xpos, false )
        end if;
    else
        if xpos=1 then
            trf := [0,pt,1,1,1,q_degree[1],0,q_degree[2],q_degree[3]];
            blowing_upy2( vf2, q_degree, 1, [ op(transformations), trf ], xpos, true )
        else
            [0,pt,1,-1,1,q_degree[1],0,q_degree[2],q_degree[3]];
            blowing_upy2( vf2, q_degree, -1, [ op(transformations), trf ], xpos, true )
        end if;
    end if;
    if save_all then openfile( terminal, result_file ) end if;
    writef( "---------------------------------------\n" );
    writef( "Back to level %d\n",lvl-1 );
    openfile( terminal );
end proc:
\end{lstlisting}

\subsection{Saving the routines in a library}

All global variables and routines are saved in a library "blowup.m".

\begin{lstlisting}[name=blowup]
save( assoc_points, newton_diagram, GAMMA_1, quasi_degree, center_type1,
    center_type2, power, sort_set_pos, sort_set_neg, put_separatrice,
    append_trajectors, split_trajectors, make_transformation, write_trajectors,
    manifold_blow_up, manifold2_blow_up, center_manifold_blow_up, blowing_up,
    blow_upx, blow_upy, sing_eigenv, blowing_upy2, blowing_up2,
    ManXPos, ManXNeg, NumXPos, NumXNeg, NumberY, ManYPos, ManYNeg,

    "blowup.m" );
\end{lstlisting}

\end{document}
